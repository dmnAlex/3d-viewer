\documentclass[12pt, a4paper]{article}

\usepackage[utf8]{inputenc}
\usepackage[T2A]{fontenc}
\usepackage[english,russian]{babel}
\usepackage{indentfirst}


\title{3DViewer v2.0}
\date{}

\begin{document}

\maketitle
\tableofcontents

\pagebreak


\section*{Описание}

Приложение для просмотра 3D моделей в каркасном виде, написанное в парадигме объектно-ориентированного программирования.


\section{Сборка и установка}

Для сборки и запуска программы необходимо установить библиотеку \textbf{QT4 или новее}.

Для формирования настоящего руководства в формате PDF необходимо установить пакеты для работы с файлами формата .tex и пакет для кириллицы.

Сборка осущетствляется в папке с исходными файлами с помощью команд \textbf{make}, \textbf{make all} или \textbf{make install}.

Исполняемый файл \textbf{3DViewer\_v2\_0} после сборки будет находиться в папке \textbf{build}.


\section{Тестирование}

Unit-тестирование загрузчика файлов моделей и аффинных преобразований программы производится с помощью библиотеки \textbf{gtest} при вызове команды \textbf{make tests}.
Для просмотра отчета о покрытии тестируемого кода - \textbf{make gcov\_report}.


\section{Удаление}
Удаление программы производится с помощью команды \textbf{make uninstall}, которая удалит каталог \textbf{build} целиком.


\section{Возможности и компоненты программы}

Программа позволяет загружать каркасную модель из файла формата \textbf{.obj} c поддержкой только списка вершин и поверхностей.
Загруженная модель может перемещаться на заданное расстояние, поворачиваться на заданный угол относительно осей XYZ, а также масштабироваться с помощью интерактивных "ползунков".
Программа позволяет настраивать тип проекции (параллельная или центральная), тип (сплошной или пунктирный), цвет и толщину ребер, а также способ отображения вершин (круг или квадрат), их цвет и размер.
В дополнение программа позволять выбирать цвет фона, а выбранные настройки сохраняются между перезапусками программы. Имеется возможность сохранения полученных в ходе трансформаций изображений в файл форматов \textbf{.bmp} и \textbf{.jpeg}.
При необходимости имеется возможность записи трансформаций изображения в течение 5-секундного интервала в \textbf{GIF-анимацию} с параметрами \textbf{640x480, 10fps}.

В программе присутствуют следующие элементы управления и вывода информации:
\begin{itemize}
    \item Кнопка выбора файла с моделью и поле для вывода его названия.
    \item Зона визуализации каркасной модели.
    \item Слайдер и поле ввода для перемещения модели.
    \item Слайдер и поле ввода для поворота модели.
    \item Слайдер и поле ввода для масштабирования модели.
    \item Поле с информацией о загруженной модели (название файла, количество вершин и ребер).
    \item Поле выбора типа проекции.
    \item Поле настройки отображения ребер.
    \item Поле настройки отображения вершин.
    \item Кнопки выбора цвета отображения для ребер, вершин и фона.
    \item Кнопки для сохранения изображения в файл и записи скринкаста (GIF).
    \item Поле выбора формата для сохранения изображения в файл (.bmp или .jpeg).
\end{itemize}

\end{document}
